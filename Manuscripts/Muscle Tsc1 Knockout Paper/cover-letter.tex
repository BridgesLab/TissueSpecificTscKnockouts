\documentclass[a4paper,10pt]{letter}
\usepackage{times}
\usepackage{wallpaper}
\ULCornerWallPaper{1}{NS-letterhead.pdf}

% Some of the article customisations are relevant for this class

\name{Dave Bridges} % To be used for the return address on the envelope
\signature{Dave Bridges, Ph.D. \\ Assistant Professor \\ Department of Nutritional Sciences  \\ University of Michigan} % Goes after the closing (ie at the end of the letter, with space for a signature)
\address{}
% Alternatively, these may be set on an individual basis within each letter environment.

%\makelabels % this command prints envelope labels on the final page of the document

\begin{document}
\begin{letter}{}

\opening{} % eg Hello.

We are pleased to submit our research article, “Skeletal Muscle mTORC1 Activation Increases Energy Expenditure and Reduces Longevity in Mice” for consideration for publication in Molecular Metabolism. 

Our manuscript reports several novel research findings that advance our understanding of the mechanisms contributing to energy homeostasis. In particular, our work demonstrates an important role for skeletal muscle-specific mechanistic target of rapamycin complex 1 (mTORC1) activity in the regulation of energy expenditure and aging.   We demonstrate that skeletal muscle mTORC1 activation increases energy expenditure and does so, we propose, in part, by a mechanism involving SR/ER uncoupling. We also report that chronic mTORC1 activation in skeletal muscle results in accelerated aging and early death.  Our findings support the hypothesis that activation of mTORC1 and its downstream targets, specifically in skeletal muscle, are important regulators of thermogenesis, and point to a role for mTORC1 in stimulating mechanisms of energy expenditure in response to caloric overload. This is a novel way of thinking about mTORC1, but since it is activated by elevated nutrient status, it is reasonable that it may also respond to these stimuli by dissipating excess energy.  Our manuscript also raises the important question of whether the positive effects of skeletal muscle mTORC1 activation (i.e., increased energy expenditure, reduced adiposity) can be separated from the negative effects of chronic mTORC1 activation (i.e., accelerated aging and early death) and leveraged in the fight against obesity.

We believe that the research findings we report are in line with the scientific mission of Molecular Metabolism. Our manuscript is not published elsewhere, nor is it under consideration for publication at any other journal. All authors contributed fairly to the finished body of work and all have approved the version of the manuscript we are submitting for your consideration. 








% This conference is specifically designed to foster cross talk between clinical care providers and basic science researchers working to alleviate obesity and its related comorbidities. This conference will be especially valuable for me because:
% •    I specifically want to focus on translational medicine in my work as a scientist. I’m trying to do this in my dissertation work by incorporating dietary and patient chart data to understand the dietary practices women are engaging in during their pregnancies and use it to fuel a mechanistic research question in animal models.
% •    I hope to improve my animal research methodologies to ask more relevant questions to further improve clinical care for women who are dealing with nutritionally complicated diagnoses during their pregnancies. The multiple talks and panel that are geared toward choosing the right model system and animal (a researcher from Jackson laboratories is slated to speak) for research that will inform future clinical practices will be valuable as I will increase my efficiency in study design.
% •    This meeting is geared toward early career professionals and students which will help to extend my professional network and include more scientists and collaborators that work in translational obesity care and research.
% •    I am especially interested in learning grant writing for translational research and there is a workshop session devoted to this during this meeting.

\closing{Regards,} % eg Regards,

%\cc{} % people this letter is cc-ed to
%\encl{} % list of anything enclosed
%\ps{} % any post scriptums. ``PS'' labels must be put in manually

\end{letter}
\end{document}
