\PassOptionsToPackage{unicode=true}{hyperref} % options for packages loaded elsewhere
\PassOptionsToPackage{hyphens}{url}
%
\documentclass[]{article}
\usepackage{lmodern}
\usepackage{amssymb,amsmath}
\usepackage{ifxetex,ifluatex}
\usepackage{fixltx2e} % provides \textsubscript
\ifnum 0\ifxetex 1\fi\ifluatex 1\fi=0 % if pdftex
  \usepackage[T1]{fontenc}
  \usepackage[utf8]{inputenc}
  \usepackage{textcomp} % provides euro and other symbols
\else % if luatex or xelatex
  \usepackage{unicode-math}
  \defaultfontfeatures{Ligatures=TeX,Scale=MatchLowercase}
\fi
% use upquote if available, for straight quotes in verbatim environments
\IfFileExists{upquote.sty}{\usepackage{upquote}}{}
% use microtype if available
\IfFileExists{microtype.sty}{%
\usepackage[]{microtype}
\UseMicrotypeSet[protrusion]{basicmath} % disable protrusion for tt fonts
}{}
\IfFileExists{parskip.sty}{%
\usepackage{parskip}
}{% else
\setlength{\parindent}{0pt}
\setlength{\parskip}{6pt plus 2pt minus 1pt}
}
\usepackage{hyperref}
\hypersetup{
            pdftitle={Evaluation of Energy Intake from BXD Datasets},
            pdfauthor={Dave Bridges},
            pdfborder={0 0 0},
            breaklinks=true}
\urlstyle{same}  % don't use monospace font for urls
\usepackage[margin=1in]{geometry}
\usepackage{color}
\usepackage{fancyvrb}
\newcommand{\VerbBar}{|}
\newcommand{\VERB}{\Verb[commandchars=\\\{\}]}
\DefineVerbatimEnvironment{Highlighting}{Verbatim}{commandchars=\\\{\}}
% Add ',fontsize=\small' for more characters per line
\usepackage{framed}
\definecolor{shadecolor}{RGB}{248,248,248}
\newenvironment{Shaded}{\begin{snugshade}}{\end{snugshade}}
\newcommand{\AlertTok}[1]{\textcolor[rgb]{0.94,0.16,0.16}{#1}}
\newcommand{\AnnotationTok}[1]{\textcolor[rgb]{0.56,0.35,0.01}{\textbf{\textit{#1}}}}
\newcommand{\AttributeTok}[1]{\textcolor[rgb]{0.77,0.63,0.00}{#1}}
\newcommand{\BaseNTok}[1]{\textcolor[rgb]{0.00,0.00,0.81}{#1}}
\newcommand{\BuiltInTok}[1]{#1}
\newcommand{\CharTok}[1]{\textcolor[rgb]{0.31,0.60,0.02}{#1}}
\newcommand{\CommentTok}[1]{\textcolor[rgb]{0.56,0.35,0.01}{\textit{#1}}}
\newcommand{\CommentVarTok}[1]{\textcolor[rgb]{0.56,0.35,0.01}{\textbf{\textit{#1}}}}
\newcommand{\ConstantTok}[1]{\textcolor[rgb]{0.00,0.00,0.00}{#1}}
\newcommand{\ControlFlowTok}[1]{\textcolor[rgb]{0.13,0.29,0.53}{\textbf{#1}}}
\newcommand{\DataTypeTok}[1]{\textcolor[rgb]{0.13,0.29,0.53}{#1}}
\newcommand{\DecValTok}[1]{\textcolor[rgb]{0.00,0.00,0.81}{#1}}
\newcommand{\DocumentationTok}[1]{\textcolor[rgb]{0.56,0.35,0.01}{\textbf{\textit{#1}}}}
\newcommand{\ErrorTok}[1]{\textcolor[rgb]{0.64,0.00,0.00}{\textbf{#1}}}
\newcommand{\ExtensionTok}[1]{#1}
\newcommand{\FloatTok}[1]{\textcolor[rgb]{0.00,0.00,0.81}{#1}}
\newcommand{\FunctionTok}[1]{\textcolor[rgb]{0.00,0.00,0.00}{#1}}
\newcommand{\ImportTok}[1]{#1}
\newcommand{\InformationTok}[1]{\textcolor[rgb]{0.56,0.35,0.01}{\textbf{\textit{#1}}}}
\newcommand{\KeywordTok}[1]{\textcolor[rgb]{0.13,0.29,0.53}{\textbf{#1}}}
\newcommand{\NormalTok}[1]{#1}
\newcommand{\OperatorTok}[1]{\textcolor[rgb]{0.81,0.36,0.00}{\textbf{#1}}}
\newcommand{\OtherTok}[1]{\textcolor[rgb]{0.56,0.35,0.01}{#1}}
\newcommand{\PreprocessorTok}[1]{\textcolor[rgb]{0.56,0.35,0.01}{\textit{#1}}}
\newcommand{\RegionMarkerTok}[1]{#1}
\newcommand{\SpecialCharTok}[1]{\textcolor[rgb]{0.00,0.00,0.00}{#1}}
\newcommand{\SpecialStringTok}[1]{\textcolor[rgb]{0.31,0.60,0.02}{#1}}
\newcommand{\StringTok}[1]{\textcolor[rgb]{0.31,0.60,0.02}{#1}}
\newcommand{\VariableTok}[1]{\textcolor[rgb]{0.00,0.00,0.00}{#1}}
\newcommand{\VerbatimStringTok}[1]{\textcolor[rgb]{0.31,0.60,0.02}{#1}}
\newcommand{\WarningTok}[1]{\textcolor[rgb]{0.56,0.35,0.01}{\textbf{\textit{#1}}}}
\usepackage{longtable,booktabs}
% Fix footnotes in tables (requires footnote package)
\IfFileExists{footnote.sty}{\usepackage{footnote}\makesavenoteenv{longtable}}{}
\usepackage{graphicx,grffile}
\makeatletter
\def\maxwidth{\ifdim\Gin@nat@width>\linewidth\linewidth\else\Gin@nat@width\fi}
\def\maxheight{\ifdim\Gin@nat@height>\textheight\textheight\else\Gin@nat@height\fi}
\makeatother
% Scale images if necessary, so that they will not overflow the page
% margins by default, and it is still possible to overwrite the defaults
% using explicit options in \includegraphics[width, height, ...]{}
\setkeys{Gin}{width=\maxwidth,height=\maxheight,keepaspectratio}
\setlength{\emergencystretch}{3em}  % prevent overfull lines
\providecommand{\tightlist}{%
  \setlength{\itemsep}{0pt}\setlength{\parskip}{0pt}}
\setcounter{secnumdepth}{0}
% Redefines (sub)paragraphs to behave more like sections
\ifx\paragraph\undefined\else
\let\oldparagraph\paragraph
\renewcommand{\paragraph}[1]{\oldparagraph{#1}\mbox{}}
\fi
\ifx\subparagraph\undefined\else
\let\oldsubparagraph\subparagraph
\renewcommand{\subparagraph}[1]{\oldsubparagraph{#1}\mbox{}}
\fi

% set default figure placement to htbp
\makeatletter
\def\fps@figure{htbp}
\makeatother


\title{Evaluation of Energy Intake from BXD Datasets}
\author{Dave Bridges}
\date{January 19, 2022}

\begin{document}
\maketitle

{
\setcounter{tocdepth}{2}
\tableofcontents
}
The goal is to identify genetic determinants of energy expenditure and
of adaptive thermogenesis from BXD mice. To start we searched gene
network for energy expenditure data, ignoring those involved in exercise
physiology.

\begin{itemize}
\tightlist
\item
  \textbf{BXD\_17621} Oxygen intake over 24h on NCD at 16 w age. Also
  included light/dark. Not adjusted for lean mass. Has mean +/- SE in
  mL/kg/h. From Prinen 2014
  (\url{https://doi.org/10.1016/j.cmet.2014.04.002})
\item
  \textbf{BXD\_17618} Oxygen intake over 24h at 16 w age on HFD?, males.
  Not adjusted for lean mass. Has mean +/- SE in mL/kg/h. From Williams
  (2016). Body weight in BXD\_17559, lean mass in BXD\_17573
\item
  \textbf{BXD\_17622} Oxygen intake over 24h at 16 w age on HFD?, males.
  Not adjusted for lean mass. Has mean +/- SE in mL/kg/h. From Williams
  (2016). Body weight in BXD\_17560, lean mass in BXD\_17574
\end{itemize}

\begin{Shaded}
\begin{Highlighting}[]
\KeywordTok{library}\NormalTok{(readr)}
\NormalTok{ncd.pirinen <-}\StringTok{ }\KeywordTok{read_csv}\NormalTok{(}\StringTok{"BXD_17621.csv"}\NormalTok{, }\DataTypeTok{skip=}\DecValTok{9}\NormalTok{) }\OperatorTok\StringTok{ }
\StringTok{  }\KeywordTok{mutate}\NormalTok{(}\DataTypeTok{Diet=}\StringTok{"NCD"}\NormalTok{,}\DataTypeTok{Age=}\DecValTok{16}\NormalTok{,}\DataTypeTok{Dataset=}\StringTok{"Prinen"}\NormalTok{)}


\NormalTok{williams.ncd.ee <-}\StringTok{ }\KeywordTok{read_csv}\NormalTok{(}\StringTok{"BXD_17622.csv"}\NormalTok{, }\DataTypeTok{skip=}\DecValTok{9}\NormalTok{)}\OperatorTok\StringTok{ }\CommentTok{#may be mislabelled on genenetwork, assigned based on HFD reductions}
\StringTok{  }\KeywordTok{mutate}\NormalTok{(}\DataTypeTok{Diet=}\StringTok{"NCD"}\NormalTok{,}\DataTypeTok{Age=}\DecValTok{16}\NormalTok{,}\DataTypeTok{Dataset=}\StringTok{"Williams"}\NormalTok{)}
\NormalTok{williams.ncd.bw <-}\StringTok{ }\KeywordTok{read_csv}\NormalTok{(}\StringTok{"BXD_17559.csv"}\NormalTok{ , }\DataTypeTok{skip=}\DecValTok{9}\NormalTok{)}\OperatorTok\StringTok{ }
\StringTok{  }\KeywordTok{mutate}\NormalTok{(}\DataTypeTok{Diet=}\StringTok{"NCD"}\NormalTok{,}\DataTypeTok{Age=}\DecValTok{16}\NormalTok{,}\DataTypeTok{Dataset=}\StringTok{"Williams"}\NormalTok{)}
\NormalTok{williams.ncd.lm <-}\StringTok{ }\KeywordTok{read_csv}\NormalTok{(}\StringTok{"BXD_17573.csv"}\NormalTok{ , }\DataTypeTok{skip=}\DecValTok{9}\NormalTok{)}\OperatorTok\StringTok{ }
\StringTok{  }\KeywordTok{mutate}\NormalTok{(}\DataTypeTok{Diet=}\StringTok{"NCD"}\NormalTok{,}\DataTypeTok{Age=}\DecValTok{16}\NormalTok{,}\DataTypeTok{Dataset=}\StringTok{"Williams"}\NormalTok{)}

\NormalTok{williams.ncd <-}\StringTok{ }\KeywordTok{full_join}\NormalTok{(williams.ncd.ee,williams.ncd.bw, }\DataTypeTok{suffix=}\KeywordTok{c}\NormalTok{(}\StringTok{'_ee'}\NormalTok{,}\StringTok{'_bw'}\NormalTok{), }\DataTypeTok{by=}\KeywordTok{c}\NormalTok{(}\StringTok{"Name"}\NormalTok{,}\StringTok{"Dataset"}\NormalTok{,}\StringTok{"Diet"}\NormalTok{)) }\OperatorTok
\StringTok{  }\KeywordTok{full_join}\NormalTok{(williams.ncd.lm) }\OperatorTok
\StringTok{  }\KeywordTok{mutate}\NormalTok{(}\DataTypeTok{Value_lm =}\NormalTok{ Value,}
         \DataTypeTok{SE_lm =}\NormalTok{ SE)}

\NormalTok{williams.hfd.ee <-}\StringTok{ }\KeywordTok{read_csv}\NormalTok{(}\StringTok{"BXD_17618.csv"}\NormalTok{ , }\DataTypeTok{skip=}\DecValTok{9}\NormalTok{)}\OperatorTok\StringTok{ }\CommentTok{#may be mislabelled on genenetwork}
\StringTok{  }\KeywordTok{mutate}\NormalTok{(}\DataTypeTok{Diet=}\StringTok{"HFD"}\NormalTok{,}\DataTypeTok{Age=}\DecValTok{16}\NormalTok{,}\DataTypeTok{Dataset=}\StringTok{"Williams"}\NormalTok{)}
\NormalTok{williams.hfd.bw <-}\StringTok{ }\KeywordTok{read_csv}\NormalTok{(}\StringTok{"BXD_17560.csv"}\NormalTok{, }\DataTypeTok{skip=}\DecValTok{9}\NormalTok{)}\OperatorTok\StringTok{ }
\StringTok{  }\KeywordTok{mutate}\NormalTok{(}\DataTypeTok{Diet=}\StringTok{"HFD"}\NormalTok{,}\DataTypeTok{Age=}\DecValTok{16}\NormalTok{,}\DataTypeTok{Dataset=}\StringTok{"Williams"}\NormalTok{)}
\NormalTok{williams.hfd.lm <-}\StringTok{ }\KeywordTok{read_csv}\NormalTok{(}\StringTok{"BXD_17574.csv"}\NormalTok{, }\DataTypeTok{skip=}\DecValTok{9}\NormalTok{)}\OperatorTok\StringTok{ }
\StringTok{  }\KeywordTok{mutate}\NormalTok{(}\DataTypeTok{Diet=}\StringTok{"HFD"}\NormalTok{,}\DataTypeTok{Age=}\DecValTok{16}\NormalTok{,}\DataTypeTok{Dataset=}\StringTok{"Williams"}\NormalTok{)}

\NormalTok{williams.hfd <-}\StringTok{ }\KeywordTok{full_join}\NormalTok{(williams.hfd.ee,williams.hfd.bw, }\DataTypeTok{suffix=}\KeywordTok{c}\NormalTok{(}\StringTok{'_ee'}\NormalTok{,}\StringTok{'_bw'}\NormalTok{), }\DataTypeTok{by=}\KeywordTok{c}\NormalTok{(}\StringTok{"Name"}\NormalTok{,}\StringTok{"Dataset"}\NormalTok{,}\StringTok{"Diet"}\NormalTok{)) }\OperatorTok
\StringTok{  }\KeywordTok{full_join}\NormalTok{(williams.hfd.lm) }\OperatorTok
\StringTok{  }\KeywordTok{mutate}\NormalTok{(}\DataTypeTok{Value_lm =}\NormalTok{ Value,}
         \DataTypeTok{SE_lm =}\NormalTok{ SE)}

\NormalTok{data <-}\StringTok{ }\KeywordTok{bind_rows}\NormalTok{(}\CommentTok{#ncd.pirinen,}
\NormalTok{                  williams.ncd,}
\NormalTok{                  williams.hfd) }\OperatorTok\StringTok{ }\CommentTok{# in mL/kg/h}
\StringTok{  }\KeywordTok{mutate}\NormalTok{(}\DataTypeTok{VO2_g_min =}\NormalTok{ Value_ee}\OperatorTok{/}\DecValTok{1000}\NormalTok{) }\OperatorTok\StringTok{ }\CommentTok{#in mL/g/h}
\StringTok{  }\KeywordTok{mutate}\NormalTok{(}\DataTypeTok{VO2_min =}\NormalTok{ VO2_g_min}\OperatorTok{*}\NormalTok{Value_bw}\OperatorTok{/}\DecValTok{60}\NormalTok{) }\OperatorTok\StringTok{ }\CommentTok{# in mL/min #this seems reasonable}
\StringTok{  }\KeywordTok{mutate}\NormalTok{(}\DataTypeTok{MR_KJ_d =}\NormalTok{ VO2_min }\OperatorTok{*}\StringTok{ }\DecValTok{60} \OperatorTok{*}\StringTok{ }\DecValTok{24} \OperatorTok{/}\StringTok{ }\DecValTok{1000} \OperatorTok{*}\StringTok{ }\FloatTok{4.84} \OperatorTok{*}\StringTok{ }\FloatTok{4.184}\NormalTok{,}
         \DataTypeTok{MR_KJ_d_SE =}\NormalTok{ SE_ee}\OperatorTok{/}\DecValTok{1000}\OperatorTok{*}\NormalTok{Value_bw}\OperatorTok{/}\DecValTok{60}\OperatorTok{*}\StringTok{ }\DecValTok{60} \OperatorTok{*}\StringTok{ }\DecValTok{24} \OperatorTok{/}\StringTok{ }\DecValTok{1000} \OperatorTok{*}\StringTok{ }\FloatTok{4.84} \OperatorTok{*}\StringTok{ }\FloatTok{4.184}\NormalTok{) }\OperatorTok\StringTok{ }\CommentTok{# 60min/h x 24h/day / 1000 mL/L x 4.84 kcal/L x 4.184 kJ/kcal}
\StringTok{  }\KeywordTok{mutate}\NormalTok{(}\DataTypeTok{MR_W =}\NormalTok{ MR_KJ_d }\OperatorTok{*}\StringTok{ }\FloatTok{0.0115740741}\NormalTok{,}
         \DataTypeTok{MR_W_SE =}\NormalTok{ MR_KJ_d_SE}\OperatorTok{*}\StringTok{ }\FloatTok{0.0115740741}\NormalTok{) }\OperatorTok\StringTok{ }\CommentTok{# in Watts }
\StringTok{  }\KeywordTok{mutate}\NormalTok{(}\DataTypeTok{Diet =} \KeywordTok{relevel}\NormalTok{(}\KeywordTok{factor}\NormalTok{(Diet), }\DataTypeTok{ref=}\StringTok{"NCD"}\NormalTok{))}
\end{Highlighting}
\end{Shaded}

These data can be found in
/Users/davebrid/Documents/GitHub/TissueSpecificTscKnockouts/Other
Published Data/Systems Biology. This script was most recently updated on
Wed Jan 19 13:30:30 2022.

\hypertarget{analysis}{%
\section{Analysis}\label{analysis}}

\hypertarget{comparason-of-datasets}{%
\subsection{Comparason of Datasets}\label{comparason-of-datasets}}

\begin{Shaded}
\begin{Highlighting}[]
\KeywordTok{library}\NormalTok{(ggplot2)}
\NormalTok{data }\OperatorTok
\StringTok{  }\KeywordTok{filter}\NormalTok{(}\OperatorTok{!}\NormalTok{(}\KeywordTok{is.na}\NormalTok{(MR_W))) }\OperatorTok\StringTok{ }\CommentTok{# complete cases only}
\StringTok{  }\KeywordTok{ggplot}\NormalTok{(}\KeywordTok{aes}\NormalTok{(}\DataTypeTok{y=}\NormalTok{MR_W,}
         \DataTypeTok{x=}\NormalTok{Name,}
         \DataTypeTok{ymin=}\NormalTok{MR_W}\OperatorTok{-}\NormalTok{MR_W_SE,}
         \DataTypeTok{ymax=}\NormalTok{MR_W}\OperatorTok{-}\NormalTok{MR_W_SE,}
         \DataTypeTok{fill=}\NormalTok{Diet)) }\OperatorTok{+}
\StringTok{  }\KeywordTok{geom_bar}\NormalTok{(}\DataTypeTok{stat=}\StringTok{'identity'}\NormalTok{,}\DataTypeTok{position=}\StringTok{'dodge'}\NormalTok{) }\OperatorTok{+}
\StringTok{  }\KeywordTok{labs}\NormalTok{(}\DataTypeTok{y=}\StringTok{"Energy Expenditure (W)"}\NormalTok{,}
       \DataTypeTok{x=}\StringTok{""}\NormalTok{) }\OperatorTok{+}
\StringTok{  }\KeywordTok{theme}\NormalTok{(}\DataTypeTok{axis.text.x =} \KeywordTok{element_text}\NormalTok{(}\DataTypeTok{angle =} \DecValTok{90}\NormalTok{, }\DataTypeTok{vjust =} \FloatTok{0.5}\NormalTok{, }\DataTypeTok{hjust=}\DecValTok{1}\NormalTok{))}
\end{Highlighting}
\end{Shaded}

\includegraphics{figures/dataset-comparason-1.png}

\begin{Shaded}
\begin{Highlighting}[]
\CommentTok{#lm(Value~Name+Diet,data=data) %>% summary}
\end{Highlighting}
\end{Shaded}

\hypertarget{adjusting-for-lean-mass}{%
\subsection{Adjusting for Lean Mass}\label{adjusting-for-lean-mass}}

\begin{Shaded}
\begin{Highlighting}[]
\KeywordTok{library}\NormalTok{(ggrepel)}
\KeywordTok{ggplot}\NormalTok{(data, }\KeywordTok{aes}\NormalTok{(}\DataTypeTok{y=}\NormalTok{MR_W,}
           \DataTypeTok{x=}\NormalTok{Value_lm)) }\OperatorTok{+}
\StringTok{  }\KeywordTok{geom_point}\NormalTok{(}\KeywordTok{aes}\NormalTok{(}\DataTypeTok{col=}\NormalTok{Diet)) }\OperatorTok{+}
\StringTok{  }\KeywordTok{geom_smooth}\NormalTok{(}\DataTypeTok{method=}\StringTok{"lm"}\NormalTok{) }\OperatorTok{+}
\StringTok{  }\KeywordTok{geom_label_repel}\NormalTok{(}\DataTypeTok{data =} \KeywordTok{subset}\NormalTok{(data, (MR_W }\OperatorTok{<}\StringTok{ }\FloatTok{0.45}\OperatorTok{&}\NormalTok{Value_lm}\OperatorTok{>}\FloatTok{25.5}\NormalTok{)}\OperatorTok{|}\NormalTok{MR_W}\OperatorTok{>}\FloatTok{0.65}\OperatorTok{&}\NormalTok{Value_lm}\OperatorTok{<}\DecValTok{27}\NormalTok{),}
                   \KeywordTok{aes}\NormalTok{(}\DataTypeTok{label=}\NormalTok{Name,}
                       \DataTypeTok{col=}\NormalTok{Diet)) }\OperatorTok{+}
\StringTok{  }\KeywordTok{labs}\NormalTok{(}\DataTypeTok{y=}\StringTok{"Energy Expenditure (W)"}\NormalTok{,}
       \DataTypeTok{x=}\StringTok{"Lean Mass (g)"}\NormalTok{)}
\end{Highlighting}
\end{Shaded}

\includegraphics{figures/lean-mass-adjusting-1.png}

\begin{Shaded}
\begin{Highlighting}[]
\KeywordTok{ggplot}\NormalTok{(data, }\KeywordTok{aes}\NormalTok{(}\DataTypeTok{y=}\NormalTok{MR_W,}
           \DataTypeTok{x=}\NormalTok{Value_lm,}
           \DataTypeTok{col=}\NormalTok{Diet)) }\OperatorTok{+}
\StringTok{  }\KeywordTok{geom_point}\NormalTok{() }\OperatorTok{+}
\StringTok{  }\KeywordTok{geom_smooth}\NormalTok{(}\DataTypeTok{method=}\StringTok{"lm"}\NormalTok{) }\OperatorTok{+}
\StringTok{  }\CommentTok{#geom_label_repel(data = subset(data, (MR_W < 0.45&Value_lm>25.5)|MR_W>0.65&Value_lm<27),aes(label=Name)) +}
\StringTok{  }\KeywordTok{labs}\NormalTok{(}\DataTypeTok{y=}\StringTok{"Energy Expenditure (W)"}\NormalTok{,}
       \DataTypeTok{x=}\StringTok{"Lean Mass (g)"}\NormalTok{)}
\end{Highlighting}
\end{Shaded}

\includegraphics{figures/lean-mass-adjusting-2.png}

\begin{Shaded}
\begin{Highlighting}[]
\CommentTok{#chow only}
\KeywordTok{ggplot}\NormalTok{(data }\OperatorTok\StringTok{ }\KeywordTok{filter}\NormalTok{(Diet}\OperatorTok{==}\StringTok{"NCD"}\NormalTok{), }\KeywordTok{aes}\NormalTok{(}\DataTypeTok{y=}\NormalTok{MR_W,}
           \DataTypeTok{x=}\NormalTok{Value_lm)) }\OperatorTok{+}
\StringTok{  }\KeywordTok{geom_point}\NormalTok{() }\OperatorTok{+}
\StringTok{  }\KeywordTok{geom_smooth}\NormalTok{(}\DataTypeTok{method=}\StringTok{"lm"}\NormalTok{) }\OperatorTok{+}
\StringTok{  }\KeywordTok{geom_label_repel}\NormalTok{(}\DataTypeTok{data =} \KeywordTok{subset}\NormalTok{(data }\OperatorTok\StringTok{ }\KeywordTok{filter}\NormalTok{(Diet}\OperatorTok{==}\StringTok{"NCD"}\NormalTok{),}
\NormalTok{                                 (MR_W }\OperatorTok{<}\StringTok{ }\FloatTok{0.48}\OperatorTok{&}\NormalTok{Value_lm}\OperatorTok{>}\FloatTok{24.5}\NormalTok{)}\OperatorTok{|}\NormalTok{MR_W}\OperatorTok{>}\FloatTok{0.60}\OperatorTok{&}\NormalTok{Value_lm}\OperatorTok{<}\DecValTok{27}\NormalTok{),}
                   \KeywordTok{aes}\NormalTok{(}\DataTypeTok{label=}\NormalTok{Name)) }\OperatorTok{+}
\StringTok{  }\KeywordTok{guides}\NormalTok{(}\DataTypeTok{fill =} \KeywordTok{guide_legend}\NormalTok{(}\DataTypeTok{override.aes =} \KeywordTok{aes}\NormalTok{(}\DataTypeTok{color =} \OtherTok{NA}\NormalTok{))) }\OperatorTok{+}
\StringTok{  }\KeywordTok{labs}\NormalTok{(}\DataTypeTok{y=}\StringTok{"Energy Expenditure (W)"}\NormalTok{,}
       \DataTypeTok{x=}\StringTok{"Lean Mass (g)"}\NormalTok{)}
\end{Highlighting}
\end{Shaded}

\includegraphics{figures/lean-mass-adjusting-3.png}

\begin{Shaded}
\begin{Highlighting}[]
\NormalTok{lm.model}\FloatTok{.1}\NormalTok{ <-}\StringTok{ }\KeywordTok{lm}\NormalTok{(MR_W}\OperatorTok{~}\NormalTok{Value_lm,}\DataTypeTok{data=}\NormalTok{data }\OperatorTok\StringTok{ }\KeywordTok{filter}\NormalTok{(Diet}\OperatorTok{==}\StringTok{"NCD"}\NormalTok{)) }\CommentTok{#model built on only NCD}
\NormalTok{lm.model}\FloatTok{.2}\NormalTok{ <-}\StringTok{ }\KeywordTok{lm}\NormalTok{(MR_W}\OperatorTok{~}\NormalTok{Value_lm}\OperatorTok{+}\NormalTok{Diet,}\DataTypeTok{data=}\NormalTok{data) }\CommentTok{#model built on NCD and AT}
\KeywordTok{library}\NormalTok{(broom)}
\KeywordTok{aov}\NormalTok{(lm.model}\FloatTok{.1}\NormalTok{) }\OperatorTok\StringTok{ }\NormalTok{tidy }\OperatorTok\StringTok{ }\KeywordTok{kable}\NormalTok{(}\DataTypeTok{caption=}\StringTok{"Model 1 summary for adjusting for lean mass"}\NormalTok{)}
\end{Highlighting}
\end{Shaded}

\begin{longtable}[]{@{}lrrrrr@{}}
\caption{Model 1 summary for adjusting for lean mass}\tabularnewline
\toprule
term & df & sumsq & meansq & statistic & p.value\tabularnewline
\midrule
\endfirsthead
\toprule
term & df & sumsq & meansq & statistic & p.value\tabularnewline
\midrule
\endhead
Value\_lm & 1 & 0.041 & 0.041 & 5.08 & 0.029\tabularnewline
Residuals & 42 & 0.342 & 0.008 & NA & NA\tabularnewline
\bottomrule
\end{longtable}

\begin{Shaded}
\begin{Highlighting}[]
\KeywordTok{summary}\NormalTok{(lm.model}\FloatTok{.1}\NormalTok{) }\OperatorTok\StringTok{ }\NormalTok{tidy }\OperatorTok\StringTok{ }\KeywordTok{kable}\NormalTok{(}\DataTypeTok{caption=}\StringTok{"Model 1 coefficients for adjusting for lean mass"}\NormalTok{)}
\end{Highlighting}
\end{Shaded}

\begin{longtable}[]{@{}lrrrr@{}}
\caption{Model 1 coefficients for adjusting for lean
mass}\tabularnewline
\toprule
term & estimate & std.error & statistic & p.value\tabularnewline
\midrule
\endfirsthead
\toprule
term & estimate & std.error & statistic & p.value\tabularnewline
\midrule
\endhead
(Intercept) & 0.176 & 0.154 & 1.15 & 0.258\tabularnewline
Value\_lm & 0.014 & 0.006 & 2.25 & 0.029\tabularnewline
\bottomrule
\end{longtable}

\begin{Shaded}
\begin{Highlighting}[]
\KeywordTok{aov}\NormalTok{(lm.model}\FloatTok{.2}\NormalTok{) }\OperatorTok\StringTok{ }\NormalTok{tidy }\OperatorTok\StringTok{ }\KeywordTok{kable}\NormalTok{(}\DataTypeTok{caption=}\StringTok{"Model 2 summary for adjusting for lean mass"}\NormalTok{)}
\end{Highlighting}
\end{Shaded}

\begin{longtable}[]{@{}lrrrrr@{}}
\caption{Model 2 summary for adjusting for lean mass}\tabularnewline
\toprule
term & df & sumsq & meansq & statistic & p.value\tabularnewline
\midrule
\endfirsthead
\toprule
term & df & sumsq & meansq & statistic & p.value\tabularnewline
\midrule
\endhead
Value\_lm & 1 & 0.106 & 0.106 & 17.2 & 0\tabularnewline
Diet & 1 & 0.095 & 0.095 & 15.5 & 0\tabularnewline
Residuals & 84 & 0.519 & 0.006 & NA & NA\tabularnewline
\bottomrule
\end{longtable}

\begin{Shaded}
\begin{Highlighting}[]
\KeywordTok{summary}\NormalTok{(lm.model}\FloatTok{.2}\NormalTok{) }\OperatorTok\StringTok{ }\NormalTok{tidy }\OperatorTok\StringTok{ }\KeywordTok{kable}\NormalTok{(}\DataTypeTok{caption=}\StringTok{"Model 2 coefficients for adjusting for lean mass"}\NormalTok{)}
\end{Highlighting}
\end{Shaded}

\begin{longtable}[]{@{}lrrrr@{}}
\caption{Model 2 coefficients for adjusting for lean
mass}\tabularnewline
\toprule
term & estimate & std.error & statistic & p.value\tabularnewline
\midrule
\endfirsthead
\toprule
term & estimate & std.error & statistic & p.value\tabularnewline
\midrule
\endhead
(Intercept) & 0.194 & 0.096 & 2.02 & 0.047\tabularnewline
Value\_lm & 0.013 & 0.004 & 3.43 & 0.001\tabularnewline
DietHFD & 0.067 & 0.017 & 3.93 & 0.000\tabularnewline
\bottomrule
\end{longtable}

\begin{Shaded}
\begin{Highlighting}[]
\NormalTok{data <-}\StringTok{ }\NormalTok{data }\OperatorTok
\StringTok{  }\KeywordTok{mutate}\NormalTok{(}\DataTypeTok{MR_predicted =} \KeywordTok{predict}\NormalTok{(lm.model}\FloatTok{.1}\NormalTok{, }\DataTypeTok{newdata =} \KeywordTok{list}\NormalTok{(}\DataTypeTok{Value_lm=}\NormalTok{Value_lm))) }\OperatorTok
\StringTok{  }\KeywordTok{mutate}\NormalTok{(}\DataTypeTok{MR_resid =}\NormalTok{ MR_W}\OperatorTok{-}\NormalTok{MR_predicted) }

\NormalTok{data }\OperatorTok
\StringTok{  }\KeywordTok{filter}\NormalTok{(}\OperatorTok{!}\KeywordTok{is.na}\NormalTok{(MR_W)) }\OperatorTok\StringTok{ }\CommentTok{# complete cases only}
\StringTok{  }\KeywordTok{ggplot}\NormalTok{(}\KeywordTok{aes}\NormalTok{(}\DataTypeTok{y=}\NormalTok{MR_resid,}
         \DataTypeTok{x=}\KeywordTok{reorder}\NormalTok{(Name,}\OperatorTok{-}\NormalTok{MR_W),}
         \DataTypeTok{ymin=}\NormalTok{MR_resid}\OperatorTok{-}\NormalTok{MR_W_SE,}
         \DataTypeTok{ymax=}\NormalTok{MR_resid}\OperatorTok{-}\NormalTok{MR_W_SE,}
         \DataTypeTok{fill=}\NormalTok{Diet)) }\OperatorTok{+}
\StringTok{  }\CommentTok{#geom_label_repel(label=Name) +}
\StringTok{  }\KeywordTok{geom_bar}\NormalTok{(}\DataTypeTok{stat=}\StringTok{'identity'}\NormalTok{,}\DataTypeTok{position=}\StringTok{'dodge'}\NormalTok{) }\OperatorTok{+}
\StringTok{  }\KeywordTok{labs}\NormalTok{(}\DataTypeTok{y=}\StringTok{"EE Observed - EE Predicted (W)"}\NormalTok{) }\OperatorTok{+}
\StringTok{  }\KeywordTok{theme}\NormalTok{(}\DataTypeTok{axis.text.x =} \KeywordTok{element_text}\NormalTok{(}\DataTypeTok{angle =} \DecValTok{90}\NormalTok{, }\DataTypeTok{vjust =} \FloatTok{0.5}\NormalTok{, }\DataTypeTok{hjust=}\DecValTok{1}\NormalTok{))}
\end{Highlighting}
\end{Shaded}

\includegraphics{figures/lean-mass-adjusting-4.png}

based on this modelling after adjusting for lean mass, HFD increases
thermogenesis by
\texttt{(coef(lm.model.2){[}"(Intercept)"{]}-coef(lm.model.2){[}"DietHFD"{]})/coef(lm.model.2){[}"(Intercept)"{]}*100}\%.

\hypertarget{adaptive-thermogenesis}{%
\subsection{Adaptive Thermogenesis}\label{adaptive-thermogenesis}}

Defined as lean mass adjusted VO2 from HFD - NCD

\begin{Shaded}
\begin{Highlighting}[]
\NormalTok{data.wide <-}
\StringTok{  }\NormalTok{data }\OperatorTok
\StringTok{  }\KeywordTok{select}\NormalTok{(Value_lm,Value_bw, MR_W, MR_W_SE, Name,Diet) }\OperatorTok
\StringTok{  }\KeywordTok{pivot_wider}\NormalTok{(}\DataTypeTok{names_from=}\NormalTok{Diet,}\DataTypeTok{id_cols=}\NormalTok{Name,}\DataTypeTok{values_from=}\KeywordTok{c}\NormalTok{(Value_lm,Value_bw, MR_W,MR_W_SE)) }\OperatorTok
\StringTok{  }\KeywordTok{mutate}\NormalTok{(}\DataTypeTok{AT =}\NormalTok{ MR_W_HFD }\OperatorTok{-}\StringTok{ }\NormalTok{MR_W_NCD,}
         \DataTypeTok{AT_SE =} \KeywordTok{sqrt}\NormalTok{((MR_W_SE_NCD}\OperatorTok{/}\NormalTok{MR_W_NCD)}\OperatorTok{^}\DecValTok{2}\OperatorTok{+}\NormalTok{(MR_W_SE_HFD}\OperatorTok{/}\NormalTok{MR_W_HFD)}\OperatorTok{^}\DecValTok{2}\NormalTok{)}\OperatorTok{*}\NormalTok{AT,}
         \DataTypeTok{Wt.Gain =}\NormalTok{ Value_bw_HFD}\OperatorTok{-}\NormalTok{Value_bw_NCD)}

\NormalTok{data.wide }\OperatorTok
\StringTok{  }\KeywordTok{filter}\NormalTok{(}\OperatorTok{!}\KeywordTok{is.na}\NormalTok{(AT)) }\OperatorTok\StringTok{ }\CommentTok{# complete cases only}
\StringTok{  }\KeywordTok{ggplot}\NormalTok{(}\KeywordTok{aes}\NormalTok{(}\DataTypeTok{y=}\NormalTok{AT,}
         \DataTypeTok{x=}\KeywordTok{reorder}\NormalTok{(Name,}\OperatorTok{-}\NormalTok{AT),}
         \DataTypeTok{ymin=}\NormalTok{AT}\OperatorTok{-}\NormalTok{AT_SE,}
         \DataTypeTok{ymax=}\NormalTok{AT}\OperatorTok{+}\NormalTok{AT_SE)) }\OperatorTok{+}
\StringTok{  }\KeywordTok{geom_bar}\NormalTok{(}\DataTypeTok{stat=}\StringTok{'identity'}\NormalTok{,}\DataTypeTok{position=}\StringTok{'dodge'}\NormalTok{) }\OperatorTok{+}
\StringTok{    }\KeywordTok{geom_errorbar}\NormalTok{() }\OperatorTok{+}
\StringTok{  }\KeywordTok{labs}\NormalTok{(}\DataTypeTok{y=}\StringTok{"HFD-Induced Adaptive Thermogenesis (W)"}\NormalTok{,}
       \DataTypeTok{x=}\StringTok{""}\NormalTok{) }\OperatorTok{+}
\StringTok{  }\KeywordTok{theme}\NormalTok{(}\DataTypeTok{axis.text.x =} \KeywordTok{element_text}\NormalTok{(}\DataTypeTok{angle =} \DecValTok{90}\NormalTok{, }\DataTypeTok{vjust =} \FloatTok{0.5}\NormalTok{, }\DataTypeTok{hjust=}\DecValTok{1}\NormalTok{))  }
\end{Highlighting}
\end{Shaded}

\includegraphics{figures/adaptive-thermogenesis-1.png}

\hypertarget{thermogenesis-on-ncd-as-a-predictor-of-weight-gain}{%
\subsubsection{Thermogenesis on NCD as a Predictor of Weight
Gain}\label{thermogenesis-on-ncd-as-a-predictor-of-weight-gain}}

\begin{Shaded}
\begin{Highlighting}[]
\NormalTok{data.wide }\OperatorTok
\StringTok{  }\KeywordTok{ggplot}\NormalTok{(}\KeywordTok{aes}\NormalTok{(}\DataTypeTok{y=}\NormalTok{Wt.Gain,}
             \DataTypeTok{x=}\NormalTok{MR_W_NCD)) }\OperatorTok{+}
\StringTok{  }\KeywordTok{labs}\NormalTok{(}\DataTypeTok{y=}\StringTok{"Weight Gained on HFD"}\NormalTok{,}
       \DataTypeTok{x=}\StringTok{"Energy Expenditure on NCD (W)"}\NormalTok{) }\OperatorTok{+}
\StringTok{  }\KeywordTok{geom_point}\NormalTok{() }\OperatorTok{+}
\StringTok{  }\KeywordTok{geom_smooth}\NormalTok{(}\DataTypeTok{method=}\StringTok{"lm"}\NormalTok{)}
\end{Highlighting}
\end{Shaded}

\includegraphics{figures/thermogenesis-weight-1.png}

\begin{Shaded}
\begin{Highlighting}[]
\KeywordTok{lm}\NormalTok{(Wt.Gain}\OperatorTok{~}\NormalTok{MR_W_NCD, }\DataTypeTok{data=}\NormalTok{data.wide) }\OperatorTok\StringTok{ }\NormalTok{glance }\OperatorTok\StringTok{ }\KeywordTok{kable}\NormalTok{(}\DataTypeTok{caption=}\StringTok{"Summary of relationship between energy expenditure and diet-induced weight gain"}\NormalTok{)}
\end{Highlighting}
\end{Shaded}

\begin{longtable}[]{@{}rrrrrrrrrrrr@{}}
\caption{Summary of relationship between energy expenditure and
diet-induced weight gain}\tabularnewline
\toprule
\begin{minipage}[b]{0.08\columnwidth}\raggedleft
r.squared\strut
\end{minipage} & \begin{minipage}[b]{0.11\columnwidth}\raggedleft
adj.r.squared\strut
\end{minipage} & \begin{minipage}[b]{0.05\columnwidth}\raggedleft
sigma\strut
\end{minipage} & \begin{minipage}[b]{0.08\columnwidth}\raggedleft
statistic\strut
\end{minipage} & \begin{minipage}[b]{0.06\columnwidth}\raggedleft
p.value\strut
\end{minipage} & \begin{minipage}[b]{0.02\columnwidth}\raggedleft
df\strut
\end{minipage} & \begin{minipage}[b]{0.05\columnwidth}\raggedleft
logLik\strut
\end{minipage} & \begin{minipage}[b]{0.03\columnwidth}\raggedleft
AIC\strut
\end{minipage} & \begin{minipage}[b]{0.03\columnwidth}\raggedleft
BIC\strut
\end{minipage} & \begin{minipage}[b]{0.07\columnwidth}\raggedleft
deviance\strut
\end{minipage} & \begin{minipage}[b]{0.09\columnwidth}\raggedleft
df.residual\strut
\end{minipage} & \begin{minipage}[b]{0.04\columnwidth}\raggedleft
nobs\strut
\end{minipage}\tabularnewline
\midrule
\endfirsthead
\toprule
\begin{minipage}[b]{0.08\columnwidth}\raggedleft
r.squared\strut
\end{minipage} & \begin{minipage}[b]{0.11\columnwidth}\raggedleft
adj.r.squared\strut
\end{minipage} & \begin{minipage}[b]{0.05\columnwidth}\raggedleft
sigma\strut
\end{minipage} & \begin{minipage}[b]{0.08\columnwidth}\raggedleft
statistic\strut
\end{minipage} & \begin{minipage}[b]{0.06\columnwidth}\raggedleft
p.value\strut
\end{minipage} & \begin{minipage}[b]{0.02\columnwidth}\raggedleft
df\strut
\end{minipage} & \begin{minipage}[b]{0.05\columnwidth}\raggedleft
logLik\strut
\end{minipage} & \begin{minipage}[b]{0.03\columnwidth}\raggedleft
AIC\strut
\end{minipage} & \begin{minipage}[b]{0.03\columnwidth}\raggedleft
BIC\strut
\end{minipage} & \begin{minipage}[b]{0.07\columnwidth}\raggedleft
deviance\strut
\end{minipage} & \begin{minipage}[b]{0.09\columnwidth}\raggedleft
df.residual\strut
\end{minipage} & \begin{minipage}[b]{0.04\columnwidth}\raggedleft
nobs\strut
\end{minipage}\tabularnewline
\midrule
\endhead
\begin{minipage}[t]{0.08\columnwidth}\raggedleft
0.509\strut
\end{minipage} & \begin{minipage}[t]{0.11\columnwidth}\raggedleft
0.498\strut
\end{minipage} & \begin{minipage}[t]{0.05\columnwidth}\raggedleft
3.65\strut
\end{minipage} & \begin{minipage}[t]{0.08\columnwidth}\raggedleft
43.6\strut
\end{minipage} & \begin{minipage}[t]{0.06\columnwidth}\raggedleft
0\strut
\end{minipage} & \begin{minipage}[t]{0.02\columnwidth}\raggedleft
1\strut
\end{minipage} & \begin{minipage}[t]{0.05\columnwidth}\raggedleft
-118\strut
\end{minipage} & \begin{minipage}[t]{0.03\columnwidth}\raggedleft
243\strut
\end{minipage} & \begin{minipage}[t]{0.03\columnwidth}\raggedleft
248\strut
\end{minipage} & \begin{minipage}[t]{0.07\columnwidth}\raggedleft
558\strut
\end{minipage} & \begin{minipage}[t]{0.09\columnwidth}\raggedleft
42\strut
\end{minipage} & \begin{minipage}[t]{0.04\columnwidth}\raggedleft
44\strut
\end{minipage}\tabularnewline
\bottomrule
\end{longtable}

\hypertarget{adaptive-thermogenesis-vs-weight-gain}{%
\subsubsection{Adaptive Thermogenesis vs Weight
Gain}\label{adaptive-thermogenesis-vs-weight-gain}}

\begin{Shaded}
\begin{Highlighting}[]
\NormalTok{data.wide }\OperatorTok
\StringTok{  }\KeywordTok{ggplot}\NormalTok{(}\KeywordTok{aes}\NormalTok{(}\DataTypeTok{y=}\NormalTok{Wt.Gain,}
             \DataTypeTok{x=}\NormalTok{AT)) }\OperatorTok{+}
\StringTok{  }\KeywordTok{labs}\NormalTok{(}\DataTypeTok{y=}\StringTok{"Weight Gained on HFD"}\NormalTok{,}
       \DataTypeTok{x=}\StringTok{"Adaptive Thermogenesis (W)"}\NormalTok{) }\OperatorTok{+}
\StringTok{  }\KeywordTok{geom_point}\NormalTok{() }\OperatorTok{+}
\StringTok{  }\KeywordTok{geom_smooth}\NormalTok{(}\DataTypeTok{method=}\StringTok{"lm"}\NormalTok{)}
\end{Highlighting}
\end{Shaded}

\includegraphics{figures/adaptive-thermogenesis-weight-1.png}

\begin{Shaded}
\begin{Highlighting}[]
\KeywordTok{lm}\NormalTok{(Wt.Gain}\OperatorTok{~}\NormalTok{AT, }\DataTypeTok{data=}\NormalTok{data.wide) }\OperatorTok\StringTok{ }\NormalTok{glance }\OperatorTok\StringTok{ }\KeywordTok{kable}\NormalTok{(}\DataTypeTok{caption=}\StringTok{"Summary of relationship between energy expenditure and diet-induced weight gain"}\NormalTok{)}
\end{Highlighting}
\end{Shaded}

\begin{longtable}[]{@{}rrrrrrrrrrrr@{}}
\caption{Summary of relationship between energy expenditure and
diet-induced weight gain}\tabularnewline
\toprule
\begin{minipage}[b]{0.08\columnwidth}\raggedleft
r.squared\strut
\end{minipage} & \begin{minipage}[b]{0.11\columnwidth}\raggedleft
adj.r.squared\strut
\end{minipage} & \begin{minipage}[b]{0.05\columnwidth}\raggedleft
sigma\strut
\end{minipage} & \begin{minipage}[b]{0.08\columnwidth}\raggedleft
statistic\strut
\end{minipage} & \begin{minipage}[b]{0.06\columnwidth}\raggedleft
p.value\strut
\end{minipage} & \begin{minipage}[b]{0.02\columnwidth}\raggedleft
df\strut
\end{minipage} & \begin{minipage}[b]{0.05\columnwidth}\raggedleft
logLik\strut
\end{minipage} & \begin{minipage}[b]{0.03\columnwidth}\raggedleft
AIC\strut
\end{minipage} & \begin{minipage}[b]{0.03\columnwidth}\raggedleft
BIC\strut
\end{minipage} & \begin{minipage}[b]{0.07\columnwidth}\raggedleft
deviance\strut
\end{minipage} & \begin{minipage}[b]{0.09\columnwidth}\raggedleft
df.residual\strut
\end{minipage} & \begin{minipage}[b]{0.04\columnwidth}\raggedleft
nobs\strut
\end{minipage}\tabularnewline
\midrule
\endfirsthead
\toprule
\begin{minipage}[b]{0.08\columnwidth}\raggedleft
r.squared\strut
\end{minipage} & \begin{minipage}[b]{0.11\columnwidth}\raggedleft
adj.r.squared\strut
\end{minipage} & \begin{minipage}[b]{0.05\columnwidth}\raggedleft
sigma\strut
\end{minipage} & \begin{minipage}[b]{0.08\columnwidth}\raggedleft
statistic\strut
\end{minipage} & \begin{minipage}[b]{0.06\columnwidth}\raggedleft
p.value\strut
\end{minipage} & \begin{minipage}[b]{0.02\columnwidth}\raggedleft
df\strut
\end{minipage} & \begin{minipage}[b]{0.05\columnwidth}\raggedleft
logLik\strut
\end{minipage} & \begin{minipage}[b]{0.03\columnwidth}\raggedleft
AIC\strut
\end{minipage} & \begin{minipage}[b]{0.03\columnwidth}\raggedleft
BIC\strut
\end{minipage} & \begin{minipage}[b]{0.07\columnwidth}\raggedleft
deviance\strut
\end{minipage} & \begin{minipage}[b]{0.09\columnwidth}\raggedleft
df.residual\strut
\end{minipage} & \begin{minipage}[b]{0.04\columnwidth}\raggedleft
nobs\strut
\end{minipage}\tabularnewline
\midrule
\endhead
\begin{minipage}[t]{0.08\columnwidth}\raggedleft
0.908\strut
\end{minipage} & \begin{minipage}[t]{0.11\columnwidth}\raggedleft
0.906\strut
\end{minipage} & \begin{minipage}[t]{0.05\columnwidth}\raggedleft
1.58\strut
\end{minipage} & \begin{minipage}[t]{0.08\columnwidth}\raggedleft
414\strut
\end{minipage} & \begin{minipage}[t]{0.06\columnwidth}\raggedleft
0\strut
\end{minipage} & \begin{minipage}[t]{0.02\columnwidth}\raggedleft
1\strut
\end{minipage} & \begin{minipage}[t]{0.05\columnwidth}\raggedleft
-81.5\strut
\end{minipage} & \begin{minipage}[t]{0.03\columnwidth}\raggedleft
169\strut
\end{minipage} & \begin{minipage}[t]{0.03\columnwidth}\raggedleft
174\strut
\end{minipage} & \begin{minipage}[t]{0.07\columnwidth}\raggedleft
105\strut
\end{minipage} & \begin{minipage}[t]{0.09\columnwidth}\raggedleft
42\strut
\end{minipage} & \begin{minipage}[t]{0.04\columnwidth}\raggedleft
44\strut
\end{minipage}\tabularnewline
\bottomrule
\end{longtable}

\hypertarget{session-information}{%
\section{Session Information}\label{session-information}}

\begin{Shaded}
\begin{Highlighting}[]
\KeywordTok{sessionInfo}\NormalTok{()}
\end{Highlighting}
\end{Shaded}

\begin{verbatim}
## R version 4.0.2 (2020-06-22)
## Platform: x86_64-apple-darwin17.0 (64-bit)
## Running under: macOS  10.16
## 
## Matrix products: default
## BLAS:   /Library/Frameworks/R.framework/Versions/4.0/Resources/lib/libRblas.dylib
## LAPACK: /Library/Frameworks/R.framework/Versions/4.0/Resources/lib/libRlapack.dylib
## 
## locale:
## [1] en_US.UTF-8/en_US.UTF-8/en_US.UTF-8/C/en_US.UTF-8/en_US.UTF-8
## 
## attached base packages:
## [1] stats     graphics  grDevices utils     datasets  methods   base     
## 
## other attached packages:
## [1] broom_0.7.11  ggrepel_0.9.1 ggplot2_3.3.5 readr_2.1.1   dplyr_1.0.7  
## [6] tidyr_1.1.4   knitr_1.37   
## 
## loaded via a namespace (and not attached):
##  [1] tidyselect_1.1.1 xfun_0.29        purrr_0.3.4      splines_4.0.2   
##  [5] lattice_0.20-45  colorspace_2.0-2 vctrs_0.3.8      generics_0.1.1  
##  [9] htmltools_0.5.2  yaml_2.2.1       mgcv_1.8-38      utf8_1.2.2      
## [13] rlang_0.4.12     pillar_1.6.4     glue_1.6.0       withr_2.4.3     
## [17] DBI_1.1.2        bit64_4.0.5      lifecycle_1.0.1  stringr_1.4.0   
## [21] munsell_0.5.0    gtable_0.3.0     evaluate_0.14    labeling_0.4.2  
## [25] tzdb_0.2.0       fastmap_1.1.0    parallel_4.0.2   fansi_1.0.0     
## [29] highr_0.9        Rcpp_1.0.7       backports_1.4.1  scales_1.1.1    
## [33] vroom_1.5.7      magick_2.7.3     farver_2.1.0     bit_4.0.4       
## [37] hms_1.1.1        digest_0.6.29    stringi_1.7.6    grid_4.0.2      
## [41] cli_3.1.0        tools_4.0.2      magrittr_2.0.1   tibble_3.1.6    
## [45] crayon_1.4.2     pkgconfig_2.0.3  ellipsis_0.3.2   Matrix_1.4-0    
## [49] assertthat_0.2.1 rmarkdown_2.11   rstudioapi_0.13  R6_2.5.1        
## [53] nlme_3.1-153     compiler_4.0.2
\end{verbatim}

\end{document}
